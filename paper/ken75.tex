\documentclass[11pt, letterpaper]{article}

\linespread{1.09}
\raggedbottom
\sloppy\sloppypar
\frenchspacing
\thispagestyle{empty}

\begin{document}

\section*{The state and future of Galactic archaeology}

\noindent
\textit{by} {David W. Hogg}\footnote{My affiliations are the
  \textsl{Center for Cosmology and Particle Physics, Department of
    Physics, New York University}, the \textsl{Center for Data
    Science, New York University}, the \textsl{Flatiron Institute, New
    York City}, and the \textsl{Max-Planck-Institut f\"ur Astronomie,
    Heidelberg}. This work was partially supported by the NSF (grant
  AST-1517237), NASA (grant NNX16AC70G), and the Moore--Sloan Data
  Science Environment at NYU. It is a pleasure to thank Andy Casey
  (Cambridge), Melissa Ness (MPIA), and Hans-Walter Rix (MPIA) for
  valuable comments.}

\paragraph{abstract:}
Foo, and bar.

\section{The state and traditions of this field}

What is Galactic Archaeology?

What is chemical tagging?

Spectroscopy and its discontents.

Story time!

Open science---and the lack thereof.

\section{New modes of observing}

\section{New modes of modeling}

Simple models vs computational models.

What you gain and lose in both cases.

\section{Data-driven methods}

Criticize machine learning and bolster The Cannon.

\section{The birthday paradox}

The paradox and what you can learn.

Numbers of stars; scaling.

\section{Probabilistic methods}

Pessimistic regime of chemical tagging.

\end{document}
