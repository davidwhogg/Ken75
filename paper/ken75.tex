\documentclass[11pt, letterpaper]{article}

% definitions
\newcommand{\documentname}{\textsl{Note}}

% spacings
\linespread{1.09}
\setlength{\parindent}{1.2\baselineskip}
\raggedbottom
\sloppy\sloppypar
\frenchspacing
\thispagestyle{empty}

\begin{document}

\section*{The state and future of Galactic archaeology%
\footnote{This \documentname\ is loosely based on remarks I made at
  \textsl{2016 Elizabeth and Frederick White Research Conference on
    Galactic Archaeology and Stellar Physics: Understanding the
    origins of the Galaxy and its stellar content}, held in honor of
  Ken Freeman (ANU). The meeting took place in Canberra, 2016 November 21--25.}}

\noindent
\textit{by} {David W. Hogg}%
\footnote{My affiliations are the \textsl{Center for Cosmology and
    Particle Physics, Department of Physics, New York University}, the
  \textsl{Center for Data Science, New York University}, the
  \textsl{Flatiron Institute, New York City}, and the
  \textsl{Max-Planck-Institut f\"ur Astronomie, Heidelberg}. This work
  was partially supported by the NSF (grant AST-1517237), NASA (grant
  NNX16AC70G), and the Moore--Sloan Data Science Environment at
  NYU. It is a pleasure to thank Andy Casey (Cambridge), Melissa Ness
  (MPIA), and Hans-Walter Rix (MPIA) for valuable comments.}

\paragraph{abstract:}
Foo, and bar.

\section{The goals and traditions of this field}

What is Galactic Archaeology? The goal of Galactic Archaeology is to
use the stars that we see in the Milky Way right now (and possibly
also the dust and gas) to infer the past history of star formation,
mass assembly, accretion, and orbital evolution of the Milky Way. That
is, the field would be a complete success if we could show the Milky
Way as it was---in shape, kinematics, and chemical compositions---at
various stages in its past, back to the beginning of its formation
shortly after the Big Bang.

In Galactic Archaeology, we would like to ``run back the clock'' on
galaxy formation! In some sense this should be impossible: In a
nonlinear dynamical system, it is impossible to run the clock back
through many dynamical times accurately! However, in the case of the
Milky Way, we have many strong beliefs that can help us regularize or
constrain this inverse problem. One thing we believe is that stars
have as invariants their surface chemical abundances, from shortly
after formation until today. This invariance is not exact: There are
dredge-up and accretion processes that will make adjustments (more
about this below). But we believe that there is a lot of information
in surface abundances. Another thing we believe is that stars that
were born together---in the same molecular cloud, say---will have
identical or at least similar chemical abundances (more about this
below). Another is that the Milky Way must have formed from somehow
``typical'' initial conditions and grown by dynamics that are very
close to those described by the $\Lambda$CDM model of physical
cosmology.

Please do not take this last point to be an endorsement of the idea
that we know everything about the cosmological model, the dark sector,
and the initial conditions. We don't! Indeed, my own personal
motivation for pursuing Galactic Archaeology is to put dynamical
constraints on the cosmological model, the initial conditions, and the
physical properties of the dark matter. But $\Lambda$CDM is such a
successful theory for explaining so much data on so many scales, it
must be at least a \emph{good approximation} to the true theory of
structure formation in the Universe. That is, any theory that replaces
$\Lambda$CDM must make near-identical predictions on very large
scales, and only different-in-detail predictions at small scales, in
order to be a contender. So whatever the true (or truer) theory of
physical cosmology, $\Lambda$CDM must be a close approximation to it,
and will probably be used as a computational surrogate, in the same
way that Newtonian Gravity is used as a computational surrogate for
the truer theory of General Relativity. That is, if Galactic
Archaeology is supremely successful, we will replace $\Lambda$CDM with
a better theory, and cold dark matter with a more physically rich
dark-matter theory. But that theory will probably be similar to
$\Lambda$CDM in its predictions, and probably be much harder to
simulate or compute.

That brings up an important point for the state of the field and what
follows in this \documentname: The fundamental theory of cosmology is
a \emph{computational} theory, a theory that makes predictions only
after enormously complex and computationally expensive simulations
have been run. These simulations involve brutal approximations and are
always resolution-limited. How can we make precise inferences in this
context?  Much work on Galactic Archaeology makes only indirect---or
no---use of the theory of galaxy formation. However, the part that
\emph{does} make use of theory falls into two general camps (and, of
course, many investigators sit in both camps): In one camp, the
approach is to simulate in gory detail as many Milky-Way-like galaxies
as you can, as realistically as you can (and typical projects can
simulate 4 to 20 of them!), and look at the properties of our own
Milky Way in the context of the equivalent properties of these
simulated exemplars. That is important for building intuition, but
it doesn't answer quantitative questions, or not with great precision.

In the other camp, the approach is to make analytic models (still
computational, but the computation is solving equations, not running a
simulation) that explain parts of the data, and in which true
inference can proceed. These models are usually regular (as in: no
chaotic orbits) and always simplistic (none of the phase-space
complexity of a $\Lambda$CDM simulation), but they permit parameter
estimation and hypothesis testing with all the precision that Bayes or
frequentism permits. There is a trade-off here between comparing the
Milky Way to our best beliefs (which are expressed by simulations) or
to our most tractable models (which permit precise measurements). This
isn't a resolved issue as of today. I will say more below.

What is chemical tagging?

Spectroscopy and its discontents.

Story time!

Open science---and the lack thereof.

\section{New modes of observing}

\section{New modes of modeling}

Simple models vs computational models.

What you gain and lose in both cases.

\section{Data-driven methods}

Criticize machine learning and bolster The Cannon.

\section{The birthday paradox}

The paradox and what you can learn.

Numbers of stars; scaling.

\section{Probabilistic methods}

Pessimistic regime of chemical tagging.

\end{document}
